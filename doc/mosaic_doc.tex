\documentclass[a4paper, 11pt]{article}
\usepackage{graphicx,wrapfig,subfigure,amsmath,amssymb,epsfig,bm}
\usepackage{listings,textcomp,color,geometry}
\usepackage{spverbatim}

\geometry{hmargin=2cm, vmargin=2cm}

\def\Box{\mathord{\dalemb{7.9}{8}\hbox{\hskip1pt}}}
\def\dalemb#1#2{{\vbox{\hrule height.#2pt
        \hbox{\vrule width.#2pt height#1pt \kern#1pt \vrule width.#2pt}
        \hrule height.#2pt}}}

\def\eop{\mathcal{E}}
\def\bop{\mathcal{B}}
\def\ba{\begin{eqnarray}}
\def\ea{\end{eqnarray}}
\def\be{\begin{equation}}
\def\ee{\end{equation}}
\def\tr{{\rm tr}}
\def\Var{{\rm Var}}
\def\gtorder{\mathrel{\raise.3ex\hbox{$>$}\mkern-14mu
             \lower0.6ex\hbox{$\sim$}}}
\def\ltorder{\mathrel{\raise.3ex\hbox{$<$}\mkern-14mu
             \lower0.6ex\hbox{$\sim$}}}

\def\bb{{\mathfrak b}}
\newcommand{\ellb }{\boldsymbol{\ell }}

% Personal colors defined here
\newcommand{\skn}[1]{{\color{red}#1}}
\newcommand{\TIB}[1]{{\color{blue}#1}}
\newcommand{\assume}[1]{{\bf#1}}

\begin{document}

\title{Mosaic: a code for power spectrum analysis in multiple patches}
\maketitle

\section{Intro}

Mosaic is a set of tools for computing angular power spectra in healpix and CAR pixellisation using the Master algorithm (arxiv:0105302).
It makes heavy use of healpix function (https://healpix.sourceforge.io/) and enlib function (see Sigurd Naess github: https://github.com/amaurea), and has benefited from Eiichiro Komatsu tools (https://www.mpa.mpa-garching.mpg.de/~komatsu/), SPHT are performed using libsharp (arXiv:1303.4945) so we are especially thankful to Martin Reinecke and Dag Sverre Seljebotn.
The code also allows to make comparison with NaMaster (see David Alonso github: https://github.com/damonge/).
While the bugs are all on me, special thanks for Sigurd Naess for help with the code.

\section{Installation}

The code is mostly written in python apart from the mode coupling calculation written in fortran 90. You need to compile it using f2py. 
Just create a make file in the makeFiles folder.
Then copy in your .bashrc the following lines:

\begin{spverbatim}
export PATH=${PATH}:PATH_TO_MOSAIC/mosaic/bin
export PATH=${PATH}:PATH_TO_MOSAIC/mosaic/tests
export PYTHONPATH=${PYTHONPATH}:PATH_TO_MOSAIC/mosaic/python
\end{spverbatim}


\section{Code structure}

The code is structured around four executables (located in the ${\textit bin} $ directory): 
\begin{enumerate}
\item ${\textit  iso\_generate\_sims.py }$
\item ${\textit  iso\_generate\_window\_functions.py }$
\item ${\textit  iso\_generate\_mcms.py  }$
\item ${\textit  iso\_generate\_all\_spectra.py  }$
\item ${\textit  iso\_generate\_mc\_results.py  } $
\item ${\textit  iso\_generate\_plots.py  } $
\end{enumerate}

They should be run in order.  
Two other executables:  ${\textit  iso\_generate\_plots.py  }$ and ${\textit  iso\_generate\_mc\_results.py  }$ allow for visualisation and combination of the results.

\subsection{Generation of simulations}

Gaussian simulations are generated from a camb lensed power spectrum.




\end{document}


